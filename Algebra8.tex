\documentclass{article}     
\usepackage[utf8]{inputenc} 
\usepackage{amsfonts}
\usepackage[left=4cm,right=4cm,
    top=3cm,bottom=4cm,bindingoffset=0cm]{geometry}
\sloppy
\usepackage[T2A]{fontenc}
\usepackage{amsmath}
\title{Домашнее задание по алгебре}  
\author{Родигина Анастасия, 167 группа}     
\date{7 мая  2017}   

\usepackage{graphicx}
\graphicspath{{pictures/}}
\DeclareGraphicsExtensions{.pdf,.png,.jpg}

\newcommand{\ip}[2]{(#1, #2)}
                             
\begin{document}            

\maketitle  
 \noindent \textbf{Задача 1}
\begin{center} 
\textit{Пусть $\alpha$   комплексный корень многочлена $x^3 - 3x + 1$. Представьте элемент
$$\frac{\alpha^4 -\alpha^3 +4\alpha+3}{\alpha^4 +\alpha^3 -2\alpha^2 +1}\in \mathbb{Q}(\alpha)$$
 в виде  $f (\alpha)$, где $f (x) \in \mathbb{Q}[x]$ и $deg f (x) \leq2$.}
\end{center}
Из того, что $f (\alpha)=0$, заметим, что $\alpha^3-3\alpha+1=0$. Повыражаем разные степени $\alpha$ и подставим в искомый элемент:
$$\alpha^3=3\alpha-1$$
$$\alpha^4=3\alpha^2-\alpha$$
$$\frac{\alpha^4 -\alpha^3 +4\alpha+3}{\alpha^4 +\alpha^3 -2\alpha^2 +1}=\frac{3\alpha^2-\alpha-3\alpha+1+4\alpha+3}{3\alpha^2-\alpha+3\alpha-1-2\alpha^2+1}=\frac{3\alpha^2+4}{\alpha^2+2\alpha}$$
Найдем такой элемент, который при умножении на $(\alpha^2+2\alpha)$ будет давать 1. Сделать это можно с помощью метода неопределенных коэффициентов:
$$1=(\alpha^2+2\alpha)(A\alpha^2+B\alpha+C)+(\alpha^3-3\alpha+1)(D\alpha^2+E\alpha+F)=$$$$=D\alpha^5+(A+E)\alpha^4+(2A+B+F-3D)\alpha^3+(2B+C+D-3E)\alpha^2+(2C+E-3F)\alpha + F$$
Решая систему уравнений на эти коэффициенты получаем многочлен: $-\alpha^2+\alpha+1$
Тогда:
$$\frac{3\alpha^2+4}{\alpha^2+2\alpha}=(3\alpha^2+4)(-\alpha^2+\alpha+1)=-10\alpha^2+16\alpha+1$$
\newline
\newline
\textbf{Задача 2}
\begin{center}
\textit{Найдите минимальный многочлен для числа $\sqrt{3} - \sqrt{5}$ над Q.}
\end{center}
Возьмем многочлен, корнем которого будет $\sqrt{3} - \sqrt{5}$:
$$x=\sqrt{3} - \sqrt{5}$$
$$x^2=8-2\sqrt{15};~~~~~ x^2-8=-2\sqrt{15}$$
$$x^4-16x^2+4=0$$
$$(x-\sqrt{3} + \sqrt{5})(x+\sqrt{3} - \sqrt{5})(x-\sqrt{3} - \sqrt{5})(x+\sqrt{3} + \sqrt{5})=x^4-16x^2+4$$
Заметим, что данный многочлен является неприводимым над $Q$ (т.е. не разлагается на множители над Q и является простым элементом кольца). А это значит, что он и будет минимальным (меньшей степени быть не может из выше сказанного ).
\newpage
\textbf{Задача 3}
\begin{center}
\textit{Пусть $F$   подполе в $\mathbb{C}$, полученное присоединением к $\mathbb{Q}$ всех комплексных корней многочлена
$x^4 + x^2 + 1$ (то есть F --- наименьшее подполе в $\mathbb{C}$, содержащее $\mathbb{Q}$ и все корни этого многочлена). Найдите степень расширения [F : $\mathbb{Q}$].} 
\end{center}
Запишем корни многочлена $x^4 + x^2 + 1$:
$$x_{1,2}=\pm\sqrt[3]{-1};~~~x_{3,4}=\pm(-1)^{2/3}$$
Несложно заметить, что все корни лежат в поле $Q[\sqrt[3]{-1}]$. Из этого следует, что степень расширения не более 2. Кроме того, используем тот факт, что $\sqrt[3]{-1}$ - нельзя представить в Q, из этого следует, что степень не менее 2. (Интересный факт, что именно так обосновываются комплексные числа (классы вычетов по модулю $x^2+1$ многочленов $R[x]$) - это просто так:))
\newline
\newline
 \noindent \textbf{Задача 4}
\begin{center}
\textit{Пусть F = $\mathbb{C}(x)$   поле рациональных дробей и $K = \mathbb{C}(y)$, где y = x + 1/x. Найдите степень расширения [F : K].}
\end{center}
$ xy = x(x + 1/x) \Rightarrow x^2-xy+1=0$,  где х будет корнем данного уравнения над $\mathbb{C}(y)$. Осталось проверить будет ли лежать этот х в $\mathbb{C}(y)$. (Если он н будет там лежать, тогда степень расширения будет равна 2)\\
Решим квадратное уравнение относительно х.
$$x=\frac{1\pm \sqrt{y^2-4}}{2}$$
 Этот многочлен является неприводимым, тогда степень расширения будет равна 2
 

\end{document}



